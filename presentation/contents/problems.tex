% !TeX encoding = UTF-8
% !TeX root = ../main.tex

%% ------------------------------------------------------------------------
%% Copyright (C) 2021 SJTUG
%% 
%% SJTUBeamer Example Document by SJTUG
%% 
%% SJTUBeamer Example Document is licensed under a
%% Creative Commons Attribution-NonCommercial-ShareAlike 4.0 International License.
%% 
%% You should have received a copy of the license along with this
%% work. If not, see <http://creativecommons.org/licenses/by-nc-sa/4.0/>.
%% -----------------------------------------------------------------------

\section{Research}

\subsection{Problems}

\begin{frame}{Blockchain background}
  \alert{Distributed ledger containing a time-stamped series of immutable blockchains, trustless, decentralized, proof-tampering and full traceability}
  \begin{itemize}
    \item Research approaches on blockchain-based crowdsensing:
          \begin{itemize}
            \item Evaluating time consumption and task cost of applying a blockchain-based system\cite{paper33}
            \item Blockchain-based crowdsensing quality control model\cite{paper34}
            \item Considering privacy issues\cite{paper35}
            \item Handling location privacy protection\cite{paper37} (confusion mechanism)
          \end{itemize}
  \end{itemize}
\end{frame}

%\begin{frame}{Research problems}
%\begin{enumerate}
%\item Organizational \textbf{P2P networks} (universities message system): \alert{efficiency?} (messaging time spent and memory space used) $\rightarrow$ \textbf{energy sustainability?}
%\item University network connections (graph theory): \textbf{predicting and recommending partners}: trust? $\rightarrow$ \textbf{links reliability?} \alert{(Machine Learning techniques)}
%\item \textbf{Competitive application calls}: \alert{auction theory (mechanism design)} auditability? $\rightarrow$ \textbf{meritocracy?}
%\end{enumerate}
%\begin{alertblock}{Research studies on practical applications?}
%So far, plenty of theoretical research on blockchain technologies but lack of empirical evidence on real-world practical applications (non-mature and complex technology)
%\end{alertblock}
%\end{frame}

\subsection{The privacy-preserving of crowdsensing}

\subsection{The incentive mechanism of crowdsensing}

\begin{frame}{The incentive mechanism of crowdsensing}
  \begin{itemize}
    \item Main types of incentive mechanisms:
          \begin{enumerate}
            \item \alert{Monetary-based}: distributing rewards. And two subtypes can be considered\cite{paper16}:
            	\begin{itemize}
            	\item \alert{price-decision-first} (auction theory) design optimal mechanism benefiting both requesters adn workers
            	\item \alert{upload-decision-first}: distributing rewards base on the uploaded data (quality)
          		\end{itemize}
            \item \alert{Reputation-based}: reputation framework for worker selection (algorithms)
          \end{enumerate}
    \item \textbf{Limitations}
    	\begin{enumerate}
            \item Relies on a central platform, vulnerable to target attacks
            \item Single-attribute incentive mechanisms (multifactor incentive needed)
          \end{enumerate}
     Some previous hybrid incentive mechanisms\cite{paper52} suffer of usability problems because the difficulty of hybrid data management
  \end{itemize}
\end{frame}