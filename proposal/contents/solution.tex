% !TeX encoding = UTF-8
% !TeX root = ../main.tex

%% ------------------------------------------------------------------------
%% Copyright (C) 2021 SJTUG
%% 
%% SJTUBeamer Example Document by SJTUG
%% 
%% SJTUBeamer Example Document is licensed under a
%% Creative Commons Attribution-NonCommercial-ShareAlike 4.0 International License.
%% 
%% You should have received a copy of the license along with this
%% work. If not, see <http://creativecommons.org/licenses/by-nc-sa/4.0/>.
%% -----------------------------------------------------------------------

\section{Planned solutions}

\subsection{ML for system}

\begin{frame}{Mechanism design: multifactor issue (static)}
\begin{multicols}{2}
  \begin{itemize}
    \item Based on \textbf{static} parameters:
          \begin{enumerate}
            \item \alert{Workers' bidding}
            \item \alert{Reputation}
            \item \alert{Recent data quality estimation}
          \end{enumerate}
    \item Analytic Hierarchy Process (AHP)
    	\begin{enumerate}
            \item \alert{Objective level}: winning workers
            \item \alert{Criteria level}: parameters criteria
            \item \alert{Alternative level}: workers available
          \end{enumerate}
    \end{itemize}
\end{multicols}
    \begin{exampleblock}{Multifactor worker evaluation approach (example)}
    	\begin{equation*}
      	\theta_{i}=\omega_{1} B_{i}+\omega_{2} R_{i}+\omega_{3} Q_{i} \quad\quad \text { where } \omega_{i} \geq 0 \text { and } \sum_{\omega_{i}=1}^{3} \omega_{i}=1	
    	\end{equation*}
  \end{exampleblock}
  \begin{block}{Approach: ML for system}
  	Dynamic evaluation parameters for the incentive model (ML techniques)
  \end{block}
\end{frame}

%\begin{frame}{Mechanism design: issues}
%  		\begin{enumerate}
%   			\item \textbf{How to select appropriate workers?}
%   				\begin{itemize}
%   					\item \textbf{Proposal: } decentralized architecture (blockchain technology) that lacks a single point of failure, and enhances privacy with asymmetric encryption and digital signature technology
%   				\end{itemize}
%    		\item \textbf{How to distribute the rewards to the workers?}
%  		\end{enumerate}
%  		With the help of \alert{mechanism design theory}\cite{article56} two important properties for the incentive mechanism are guaranteed:
%  		\begin{itemize}
%   					\item \textbf{Incentive quality (IC):} the truthful submission of sensing  cost is the worker's optimal bidding strategy
%   					\item \textbf{Individual rationality (IR):} the reward must compensate for the worker's cost (non-negative)
%   		\end{itemize}
%\end{frame}

\subsection{Bonus: System for ML}

\begin{frame}{System efficiency: consensus protocol}
\alert{Blockchain-based Federated Learning (FL)}
  \begin{itemize}
    \item Potential bottlenecks:
    	\begin{enumerate}
            \item Transaction throughput
            \item Lack of scalability of data
          \end{enumerate}
    \end{itemize}

  \begin{block}{Approach: System for ML}
  	Optimization of consensus protocol in FL (better performance)
  \end{block}
\end{frame}
