% !TeX encoding = UTF-8
% !TeX root = ../main.tex

%% ------------------------------------------------------------------------
%% Copyright (C) 2021 SJTUG
%% 
%% SJTUBeamer Example Document by SJTUG
%% 
%% SJTUBeamer Example Document is licensed under a
%% Creative Commons Attribution-NonCommercial-ShareAlike 4.0 International License.
%% 
%% You should have received a copy of the license along with this
%% work. If not, see <http://creativecommons.org/licenses/by-nc-sa/4.0/>.
%% -----------------------------------------------------------------------

\section{Research problems}

\subsection{The incentive mechanism}

\begin{frame}{Application to Federated Learning}
  \begin{itemize}
    \item Main types of incentive mechanisms:
          \begin{enumerate}
            \item \alert{Monetary-based}: distributing rewards. And two subtypes can be considered:%\cite{paper16}
            	\begin{itemize}
            	\item \alert{price-decision-first} (auction theory) design optimal mechanism benefiting both requesters adn workers
            	\item \alert{upload-decision-first}: distributing rewards base on the uploaded data (quality)
          		\end{itemize}
            \item \alert{Reputation-based}: reputation framework for worker selection (algorithms)
          \end{enumerate}
    \item \textbf{Limitations}
    	\begin{enumerate}
            \item Relies on a central platform (parameter server), vulnerable to target attacks
            \item Single-attribute incentive mechanisms (multifactor incentive needed)
          \end{enumerate}
     Some previous hybrid incentive mechanisms suffer of usability problems because the difficulty of hybrid data management%\cite{paper52}
  \end{itemize}
\end{frame}

\subsection{Benefits and limitations}

\begin{frame}{Benefits}
	A consortium blockchain-based incentive model for Federated Learning system is
proposed
  \begin{itemize}
  \item \textbf{Benefits of consortium blockchain technology:} 
  	\begin{itemize}
  		\item resistant to the single point of failure (system security)
  		\item cooperative management (by requesters) reduces cost and enhances the flexibility of the system (selection criteria)
  	\end{itemize}
  \item \textbf{Benefits of hybrid incentive mechanism:}
  	\begin{itemize}
  		\item encourages workers to contribute valuable data (and penalizes malicious ones)
  		\item ensures favorables short-term and long-term incentives for workers
  	\end{itemize}
  \end{itemize}
\end{frame}

\begin{frame}{Limitations}
	Further research:
  \begin{enumerate}
  \item Dynamic situation where evaluation attributes are changing
  \item Optimization of consensus protocol (better performance)
  \item Further protection of worker privacy
  \end{enumerate}
  Possible approaches:
  \begin{block}{ML for systems}
  ML applied to blockchain-based trust management system (better incentive model)
  \end{block}
  \begin{block}{Systems for ML}
  System improvement for FL (better blockchain-based model)
  \end{block}
\end{frame}