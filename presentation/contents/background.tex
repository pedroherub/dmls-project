% !TeX encoding = UTF-8
% !TeX root = ../main.tex

%% ------------------------------------------------------------------------
%% Copyright (C) 2021 SJTUG
%% 
%% SJTUBeamer Example Document by SJTUG
%% 
%% SJTUBeamer Example Document is licensed under a
%% Creative Commons Attribution-NonCommercial-ShareAlike 4.0 International License.
%% 
%% You should have received a copy of the license along with this
%% work. If not, see <http://creativecommons.org/licenses/by-nc-sa/4.0/>.
%% -----------------------------------------------------------------------

\section{Background}

\subsection{Motivation}

\begin{frame}{Motivation}
%	\begin{block}{Goal}
%        Applying ML to systems (blockchain-based models)
%      \end{block}
  \begin{itemize}
    \item Research interest mainly targeted to \alert{blockchain technology} (its application to distributed Machine Learning systems)
%    \item Research group in Department of Automation (PhD supervisors) has been recently working in blockchain-based models applied to \alert{trust management systems}
	\item Specifically, applied to data-aggregation systems (e.g. \alert{crowdsensing} in the Internet of Things (IoT) field)
	\item Could similar approach be applied for \alert{Federated Learning}?
%    \item My PhD thesis could use some of the model simulation techniques (stochastic processes) used in this research paper experimentation
%    \item And of course, it is related to a emerging subfield that could greatly contribute to the development of the Internet of Things (IoT)
  \end{itemize}
\end{frame}

\begin{frame}{Blockchain technology}
\begin{multicols}{2}
  \alert{Distributed ledger containing a time-stamped series of immutable blockchains, trustless, decentralized, proof-tampering and full traceability}
%   \begin{itemize}
%     \item Research approaches on blockchain-based crowdsensing:
%           \begin{itemize}
%             \item Evaluating time consumption and task cost of applying a blockchain-based system\cite{paper33}
%             \item Blockchain-based crowdsensing quality control model\cite{paper34}
%             \item Considering privacy issues\cite{paper35}
%             \item Handling location privacy protection\cite{paper37} (confusion mechanism)
%           \end{itemize}
%   \end{itemize}
    \begin{figure}[h]
        \centering
        \includegraphics[width=5cm]{topologyBC.PNG}
    \end{figure}
\end{multicols}
\end{frame}

\subsection{Crowdsensing}

\begin{frame}{Crowdsensing: definition}
  \begin{itemize}
  \item \textbf{Crowdsensing:} emerging paradigm of data aggregation, having a key role in data-driven applications. Specially used for getting large ammounts of IoT sensing data, by using the individual intelligent sensing devices.
  \item \textbf{Benefit:} improved data collection efficiency and reduced costs effectively
  \end{itemize}
  \begin{figure}[h]
        \centering
        \includegraphics[width=.8\textwidth]{201909-wei-figure1.jpg}
      \end{figure}
\end{frame}

\begin{frame}{Crowdsensing issues: security}
\begin{multicols}{2}
  		\begin{enumerate}
   			\item Managed and maintained \alert{centralized platforms} suffer from the single point of failure
   				% \begin{itemize}
   				% 	\item \textbf{Proposal: } decentralized architecture (blockchain technology) that lacks a single point of failure, and enhances privacy with asymmetric encryption and digital signature technology
   				% \end{itemize}
  		\end{enumerate}
  		\begin{figure}[h]
                \centering
                \includegraphics[width=5cm]{topologyFL.PNG}
                \caption{Topology of traditional FL}
            \end{figure}
  \end{multicols}
\end{frame}

\begin{frame}{Crowdsensing issues: incentive}
\begin{multicols}{2}
  		\begin{enumerate}
    		\item Encouraging workers by offering appropiate \alert{incentive mechanisms} (monetary usually) \rightarrow   \underline{auction theory} guarantees benefits for both requesters and workers but only provide short-term incentives
    			% \begin{itemize}
   				% 	\item \textbf{Proposal:} hybrid incentive mechanism, adopting \underline{mechanism design theory}, considering three factors:
   				% 	\begin{itemize}
   				% 	\item Monetary reward
   				% 	\item Reputation evaluation
   				% 	\item Data quality
   				% 	\end{itemize}
   				% \end{itemize}
                   \begin{figure}[H]
                    \centering
%                    \begin{subfigure}
                      \centering
                      \includegraphics[width=1cm]{money.PNG}
                      \caption{Monetary reward}
%                    \end{subfigure}%
%                    \begin{subfigure}
                      \centering
                      \includegraphics[width=1cm]{reputation.PNG}
                      \caption{Worker reputation}
%                    \end{subfigure}%
%                    \begin{subfigure}
                        \centering
                        \includegraphics[width=1cm]{dataQuality.PNG}
                        \caption{Data quality}
%                      \end{subfigure}
%                    \caption{Incentive mechanisms}
%                    \label{fig:incentive mechanisms}
                    \end{figure}
  		\end{enumerate}
  \end{multicols}
\end{frame}

%\section{Background}
%
%\subsection{Motivation}
%
%\begin{frame}{Motivation}
%	\begin{block}{Goal}
%        Applying ML to systems (blockchain-based models)
%      \end{block}
%  \begin{itemize}
%    \item Research line mainly targeted to \alert{blockchain technology} (its application to systems)
%    \item Research group in Department of Automation (PhD supervisors) has been recently working in blockchain-based models applied to \alert{trust management systems}
%	\item Specifically, applied to data-aggregation systems in the Internet of Things (IoT) field \alert{crowdsensing}
%	\item Could similar approach be applied for \alert{Federated Learning}?
%%    \item My PhD thesis could use some of the model simulation techniques (stochastic processes) used in this research paper experimentation
%%    \item And of course, it is related to a emerging subfield that could greatly contribute to the development of the Internet of Things (IoT)
%  \end{itemize}
%\end{frame}
%
%\subsection{Crowdsensing}
%
%\begin{frame}{Crowdsensing: definition}
%  \begin{itemize}
%  \item \textbf{Crowdsensing:} emerging paradigm of data aggregation\cite{paper1}, having a key role in data-driven applications. Specially used for getting large ammounts of IoT sensing data, by using the individual intelligent sensing devices.
%  \item \textbf{Benefit:} improved data collection efficiency and reduced costs effectively\cite{paper2}
%  \end{itemize}
%  \begin{figure}[h]
%        \centering
%        \includegraphics[width=.8\textwidth]{201909-wei-figure1.jpg}
%      \end{figure}
%\end{frame}
%
%\begin{frame}{Crowdsensing: issues}
%  		\begin{enumerate}
%   			\item Managed and maintained \alert{centralized platforms} suffer from the single point of failure
%   			\begin{figure}[h]
%                \centering
%                \includegraphics[width=8cm]{topologyFL.PNG}
%                \caption{Topology of traditional FL}
%            \end{figure}
%   				% \begin{itemize}
%   				% 	\item \textbf{Proposal: } decentralized architecture (blockchain technology) that lacks a single point of failure, and enhances privacy with asymmetric encryption and digital signature technology
%   				% \end{itemize}
%    		\item Encouraging workers by offering appropiate \alert{incentive mechanisms} (monetary usually) \rightarrow  \underline{auction theory} guarantees benefits for both requesters and workers\cite{paper15} but only provide short-term incentives
%    			% \begin{itemize}
%   				% 	\item \textbf{Proposal:} hybrid incentive mechanism, adopting \underline{mechanism design theory}, considering three factors:
%   				% 	\begin{itemize}
%   				% 	\item Monetary reward
%   				% 	\item Reputation evaluation
%   				% 	\item Data quality
%   				% 	\end{itemize}
%   				% \end{itemize}
%                   \begin{figure}[H]
%                    \centering
%                    \begin{subfigure}
%                      \centering
%                      \includegraphics[width=2cm]{money.PNG}
%                      \caption{Monetary reward}
%                    \end{subfigure}%
%                    \begin{subfigure}
%                      \centering
%                      \includegraphics[width=2cm]{reputation.PNG}
%                      \caption{Worker reputation}
%                    \end{subfigure}%
%                    \begin{subfigure}
%                        \centering
%                        \includegraphics[width=2cm]{dataQuality.PNG}
%                        \caption{Data quality}
%                      \end{subfigure}
%                    \caption{Incentive mechanisms}
%                    \label{fig:incentive mechanisms}
%                    \end{figure}
%  		\end{enumerate}
%\end{frame}
%
%%\section{Background}
%%
%%\subsection{Motivation}
%%
%%\begin{frame}{Motivation}
%%	\begin{block}{Goal}
%%        Applying ML to systems (blockchain-based models)
%%      \end{block}
%%  \begin{itemize}
%%    \item Research line mainly targeted to \alert{blockchain technology} (its application to systems)
%%    \item Research group in Department of Automation (PhD supervisors) has been recently working in blockchain-based models applied to \alert{trust management systems}
%%	\item Specifically, applied to data-aggregation systems in the Internet of Things (IoT) field \alert{crowdsensing}
%%	\item Could similar approach be applied for \alert{Federated Learning}?
%%%    \item My PhD thesis could use some of the model simulation techniques (stochastic processes) used in this research paper experimentation
%%%    \item And of course, it is related to a emerging subfield that could greatly contribute to the development of the Internet of Things (IoT)
%%  \end{itemize}
%%\end{frame}
%%
%%\subsection{Crowdsensing}
%%
%%\begin{frame}{Crowdsensing: definition}
%%  \begin{itemize}
%%  \item \textbf{Crowdsensing:} emerging paradigm of data aggregation\cite{paper1}, having a key role in data-driven applications. Specially used for getting large ammounts of IoT sensing data, by using the individual intelligent sensing devices.
%%  \item \textbf{Benefit:} improved data collection efficiency and reduced costs effectively\cite{paper2}
%%  \end{itemize}
%%  \begin{figure}[h]
%%        \centering
%%        \includegraphics[width=.8\textwidth]{201909-wei-figure1.jpg}
%%      \end{figure}
%%\end{frame}
%%
%%\begin{frame}{Crowdsensing: issues}
%%  		\begin{enumerate}
%%   			\item Managed and maintained \alert{centralized platforms} suffer from the single point of failure
%%   				\begin{itemize}
%%   					\item \textbf{Proposal: } decentralized architecture (blockchain technology) that lacks a single point of failure, and enhances privacy with asymmetric encryption and digital signature technology
%%   				\end{itemize}
%%    		\item Encouraging workers by offering appropiate \alert{incentive mechanisms} (monetary usually) \rightarrow  \underline{auction theory} guarantees benefits for both requesters and workers\cite{paper15} but only provide short-term incentives
%%    			\begin{itemize}
%%   					\item \textbf{Proposal:} hybrid incentive mechanism, adopting \underline{mechanism design theory}, considering three factors:
%%   					\begin{itemize}
%%   					\item Monetary reward
%%   					\item Reputation evaluation
%%   					\item Data quality
%%   					\end{itemize}
%%   				\end{itemize}
%%  		\end{enumerate}
%%\end{frame}
