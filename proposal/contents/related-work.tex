% !TeX encoding = UTF-8
% !TeX root = ../main.tex

%% ------------------------------------------------------------------------
%% Copyright (C) 2021 SJTUG
%% 
%% SJTUBeamer Example Document by SJTUG
%% 
%% SJTUBeamer Example Document is licensed under a
%% Creative Commons Attribution-NonCommercial-ShareAlike 4.0 International License.
%% 
%% You should have received a copy of the license along with this
%% work. If not, see <http://creativecommons.org/licenses/by-nc-sa/4.0/>.
%% -----------------------------------------------------------------------

\section{Related work}

\subsection{Blockchain-based Federated Learning}

\begin{frame}{Blockchain-based Federated Learning}
     Some previous hybrid incentive mechanisms\cite{paper52} suffer of usability problems because the difficulty of hybrid data management
\end{frame}

\begin{frame}
Advances and Open Problems in Federated Learning\cite{1912.04977}


This paper describes the defining characteristics and challenges of the federated learning setting.It highlights important practical constraints and considerations, and then enumerates a range of valuable
research directions. The goals of this paper are to highlight research problems that are of significant theoretical and practical interest, and to encourage research on problems that could have significant real-world impact
\end{frame}

\begin{frame}
Generative models for effective ML in  private, decentralized datasets\cite{1911.06679}


This document focuses on three main topics : 


-Identifying key challenges in implementing end-to-end workflows with non-inspectable data


-Proposing a methodology that allows ‘auxiliary’ generative models to resolve these challenges


-Demonstrating how privacy preserving federated generative models can be trained to high enough fidelity to discover introduced data errors matching
those encountered in real world scenarios
\end{frame}

\begin{frame}
Blockchain-based Federated Learning: A Comprehensive Survey\cite{Comprehensive_survey}


This paper conducts a comprehensive survey of the literature on blockchained FL (BCFL). First, it investigates how blockchain can be applied to federal learning from the perspective of system composition. Then, it analyzes the concrete functions of BCFL from the perspective of mechanism design and illustrate what problems blockchain addresses specifically for FL. Finally, it discusses some challenges and future research directions.
\end{frame}

\begin{frame}
Federated Learning Meets Blockchain in Edge Computing: Opportunities and Challenges\cite{nguyen_federated_2021}


This article presents an overview of the fundamental concepts and explores the opportunities of FL chain in mobile-edge computing networks in relation with his main issues in FL chain design such as communication cost, resource allocation, incentive mechanism, security and privacy protection. The key solutions and the lessons learned along with the outlooks are also discussed. Then, it investigates the applications of FL chain in popular Mobile-edge computing domains.
\end{frame}
\subsection{Incentive mechanisms applied to Federated Learning}

\begin{frame}{Incentive mechanisms applied to Federated Learning}
     Some previous hybrid incentive mechanisms\cite{paper52} suffer of usability problems because the difficulty of hybrid data management
\end{frame}
